\section{Motivation}
Black-box systems are used often in various applications. 
Probably most of the tools and software that we use everyday in our work or every-day life are black-box systems; meaning that we do not have access to their internal workings and their source code and cannot make any changes to them.
But in the end of the day, black box systems just like any other human developed systems are not perfect.
They are prone to have bugs or faults. For that reason, depending on the application, we might need to verify if they work as advertised. For some cases there are standards and authorities that perform the verification on the users' behalf, for example there are FAA certificates for aircraft and street-legality requirements for cars, but this is a luxury not available in all markets. Even on the systems that do have the standards and inspections we might need further more strict and more customized verifications.

Creating a model of the system we have in hand proves to be extremely useful. 
In addition to verification, it can be used for debugging \cite{jafar2019interactive, hybriddebugging, shang2013assisting} and for testing \cite{Walkinshaw2018TestingBlackBox, ModelBasedTesting, Papadopoulos2015, dallmeier2011automatically}.
In this paper we focus on state-based family of systems.


Black-box model inference methods are quite useful when some one decides to use COTS components and they need to inspect the components to verify if work as expected, debug them, or maybe even reverse engineer them. This is a very common scenario in the defense industry.

We propose a method to infer state models from black-box systems which can be put to various uses. An state model helps us to know the internal state of the system which we can use to predict how it will behave in near future.
% This model provides a non-linear approximation of the system
This method helps automate and accelerate model inference therefore facilitating the larger process model inference is a part of. 


What that we test our method on is a commercial UAV\footnote{Unmanned Aerial Vehicle} AutoPilot: a black-box safety critical state-based system. As a black-box its inputs and output signals are accessible. Interpreting them as numerical values id pretty straightforward. However our method is not limited to simple numerical inputs and outputs. 
With some changes it can be used for similar systems such as IoT devices, intelligent video surveillance systems, driver assisting systems, and self driving cars.
